\subsection{Spherical coordinate system}
A sphere with center $O=(x_0,y_0,z_0)$ and radius $r$ is the set of points at distance exactly $r$ from $O$.
We can describe it this by equation
\[(x-x_0)^2+(y-y_0)^2+(z-z_0)^2 = r^2\]

\centerFig{sph-0}

To describe a point on a sphere, we can either directly give its coordinates $x,y,z$ or use the spherical coordinate system: this is the system used to position locations on Earth. We use an angle $\varphi$, the latitude, which tells us how far North the point is (or South, if $\varphi<0$), and an angle $\lambda$, the longitude, which tells us how far East the point is (or West, if $\lambda<0)$. We usually take $\varphi \in [-\frac{\pi}{2},\frac{\pi}{2}]$ and $\lambda \in (-\pi,\pi]$.

\centerFig{sph-1}

If the sphere is centered at the origin, the position represented by coordinates $(\varphi,\lambda)$ is
\[(r\cos\varphi\cos\lambda, r\cos\varphi\sin\lambda, r\sin\varphi)\]
where the $z$-axis points North and the $x$-axis points towards meridian $\lambda=0$ (on Earth, the Greenwich meridian).
%Note that this also gives a parametric representation of the sphere.

The function below finds the position given angles in degrees:
\begin{lstlisting}
p3 sph(double r, double lat, double lon) {
    lat *= M_PI/180, lon *= M_PI/180;
    return {r*cos(lat)*cos(lon), r*cos(lat)*sin(lon), r*sin(lat)};
}
\end{lstlisting}

\subsection{Sphere-line intersection}
A sphere $(O,r)$ and a line $l$ have either 0, 1, or 2 intersection points.

\centerFig{sph-2}

Finding them is exactly like circle-line intersection: first compute the projection $P$ of the center onto the line, then find the intersections by moving the appropriate distance forward or backward along $l$.

\centerFig{sph-3}

This function returns the number of intersection points and places them in pair \lstinline|out| if they exist.
\begin{lstlisting}
int sphereLine(p3 o, double r, line3d l, pair<p3,p3> &out) {
    double h2 = r*r - l.sqDist(o);
    if (h2 < 0) return 0; // the line doesn't touch the sphere
    p3 p = l.proj(o); // point P
    p3 h = l.d*sqrt(h2)/abs(l.d); // vector parallel to l, of length h
    out = {p-h, p+h};
    return 1 + (h2 > 0);
}
\end{lstlisting}
