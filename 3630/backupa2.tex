\documentclass{article}
% \usepackage[utf8]{inputenc}

% usage: from Options > configure TexStudio > Build > add user command
% makeindex -s nomencl.ist -t %.nlg -o %.nls %.nlo
% Now run pdflatex -> Make Nomenclature -> pdflatex
\usepackage{nomencl}
\makenomenclature

\usepackage{tikz}
\usetikzlibrary{automata,positioning, shapes.geometric}
\usepackage{subfig}
\usepackage{pgf}
\usepackage{float}
\usepackage{titlesec}
\usepackage[nottoc]{tocbibind}
\usepackage{amsmath}
\usepackage{hyperref}
\usepackage{enumitem}
\usepackage{listings}
\usepackage{xcolor}
\usepackage{amsmath,amssymb,amsthm}
\usepackage{graphicx}
\graphicspath{./images/}



% set builder notation utility function, usage: \Set{ x\in A \given x^2 \geq 3 }
\usepackage{mathtools}
\newcommand\SetSymbol[1][]{\nonscript\:#1\vert\allowbreak\nonscript\:\mathopen{}}
\providecommand\given{} % to make it exist
\DeclarePairedDelimiterX\Set[1]\{\}{\renewcommand\given{\SetSymbol[\delimsize]}#1}

% some stuff for C++
\lstset { %
	language=C++,
	backgroundcolor=\color{black!5}, % set backgroundcolor
	basicstyle=\footnotesize,% basic font setting
	breaklines=true,
	postbreak=\mbox{\textcolor{red}{$\hookrightarrow$}\space},
	frame=single,
	columns=fullflexible
}

% TABLE OF CONTENTS, MANUALLY SET NESTING DEPTH
%\setcounter{tocdepth}{1} % show sections
%\setcounter{tocdepth}{2} % show subsections
%\setcounter{tocdepth}{3} % show subsubsections
%\setcounter{tocdepth}{4} % show paragraphs
%\setcounter{tocdepth}{5} % show subparagraphs

% VARIABLES
% --- Hide title page
%\newif\ifhidetitle
%\hidetitletrue

% --- make subsections have lettering (e.g. 1.a, 1.b)
% \renewcommand{\thesubsection}{\thesection.\alph{subsection}}

% --- make subsections have arabic numbering (e.g. 1.1, 1.2, ...)
\renewcommand{\thesubsection}{\arabic{subsection}}

% --- End of proof black square (remove if you want default hollow white square)
\renewcommand{\qedsymbol}{$\blacksquare$}

\titlespacing*{\section}
{0pt}{5.5ex plus 1ex minus .2ex}{4.3ex plus .2ex}
\titlespacing*{\subsection}
{0pt}{5.5ex plus 1ex minus .2ex}{4.3ex plus .2ex}


\begin{document}
	
% --- title, author, date all on title page, use \\ within a {} to linebreak
\title{CPSC 3630\\Assignment 1}
\author{Cody Barnson\\ ID: 001172313}
\date{1 Oct 2018}

% TITLE PAGE, example of variable usage
%\ifhidetitle
%\else

% --- no page numbers for title page and table of contents
\pagenumbering{gobble}
\maketitle
\newpage

% --- uncomment to show the table of contents
%\tableofcontents
%\listoftables
%\listoffigures
\newpage

% --- this 'fi' belongs to the \ifhidetitle \else block above
%\fi

% --- normal page numbers starting from here, can also do roman
\pagenumbering{arabic}
		
		
% ====== START HERE
		
\section*{}

\subsection{}

\subsubsection{Prove the following statement}

\paragraph{Theorem 1} The class of regular languages is closed under the intersection operation. \\

Suppose $A_1$ and $A_2$ are regular languages.  We wish to show that $A_1 \cap A_2$ is regular also.  Let FA $M_1$ recognize $A_1$, where $M_1 = (Q_1, \Sigma, \delta_1, q_1, F_1)$, and let $M_2$ recognize $A_2$, where $M_2 = (Q_2, \Sigma, \delta_2, q_2, F_2)$.

We construct FA $M$ to recognize $A_1 \cap A_2$ where $M = (Q, \Sigma, \delta, q_0, F)$.  

\begin{enumerate}
    \item $Q = \Set{(r_1, r_2) \given r_1 \in Q \; and \; r_2 \in Q_2}$
    \item $\Sigma$ is same as in $M_1$ and $M_2$.
    \item $\delta$ is transition function given as follows,
    $\forall (r_1, r_2) \in Q$ and each $a \in \Sigma$ let $\delta((r_1, r_2), a) = (\delta(r_1, a), \delta(r_2, a))$, hence $\delta$ gets a state of $M$.
    \item $q_0$ is the pair $(q_1, q_2)$.
    \item $F$ is the set of pairs in which both members $r_1 \in F_1$, $r_2 \in F_2$ are accept states of $M_1$ and $M_2$, respectively.  That is, \\ $F = \Set{(r_1, r_2) \given r_1 \in F_1 \; and \; r_2 \in F_2} = F_1 \times F_2$.
\end{enumerate}

This concludes the construction of the FA $M$ that recognizes $A_1 \cap A_2$, as desired.

\subsubsection{Let $\Sigma = \Set{a,b}$. Construct DFA's for the simpler languages and use Theorem 1 to give the state-diagram of a DFA for the language given below.}

\paragraph{$\Set{w \given w \; \text{has odd number of $a$'s and at most one $b$}}$}

\paragraph{}

From our work in question 1(a), we have a construction for the intersection of 2 languages, say $A_1$ and $A_2$.  We have some language, as given, that $A_1 \cap A_2$ recognizes.  \\

We begin by describing the DFA's $D_1$ and $D_2$ that recognize $A_1$ and $A_2$, respectively.  

First, we give $D_1$, where $D_1 = (Q_1, \Sigma, \delta_1, q_1, F_1)$, that recognizes language given by $A_1 = \Set{w \given w \; \text{has odd number of $a$'s}}$.

\begin{enumerate}
    \item $Q_1 = \Set{q_{11}, q_{12}}$
    \item $\Sigma = \Set{a,b}$
    \item $\delta_1$ is given by,
    
\begin{figure}[H]
\centering

\[
\begin{array}{c||c|c}
         & a & b \\ \hline \hline 
        q_{11} & q_{12} & q_{11} \\ \hline 
        q_{12} & q_{11} & q_{12} \\ \hline 
\end{array}
\]

\caption{}
\label{fig:1bd1}
\end{figure}


    
    \item Start state is $q_1 = q_{11}$
    \item Set of accept states is $F_1 = \Set{q_{12}}$
\end{enumerate}

Second, we give $D_2$ where $D_2 = (Q_2, \Sigma, \delta_2, q_2, F_2)$ that recognizes language $A_2$ where $A_2 = \Set{w \given w \; \text{has at most one $b$}}$.  

\begin{enumerate}
    \item $Q_2 = \Set{q_{21}, q_{22}, q_{23}}$
    \item $\Sigma = \Set{a,b}$
    \item $\delta_2$ is given by, 
    
\begin{figure}[H]
\centering

\[
\begin{array}{c||c|c}
         & a & b \\ \hline \hline 
        q_{21} & q_{21} & q_{22} \\ \hline 
        q_{22} & q_{22} & q_{23} \\ \hline 
        q_{23} & q_{23} & q_{23} \\ \hline 
\end{array}
\]

\caption{}
\label{fig:1bd2}
\end{figure}

    
    \item Start state is $q_2 = q_{21}$
    \item Set of accept states is $F_2 = \Set{q_{21}, q_{22}}$
\end{enumerate}

\paragraph{}

We construct DFA $D$ that recognizes $A_1 \cap A_2$.  $D = (Q, \Sigma, \delta, q_0, F)$.

\begin{enumerate}
    \item 
    \begin{align*}
        Q &= \Set{(r_1, r_2) \given r_1 \in Q_1 \; and \; r_2 \in Q_2} \\ &= \Set{(q_{11}, q_{21}), (q_{11}, q_{22}), (q_{11}, q_{23}), (q_{12}, q_{21}), (q_{12}, q_{22}), (q_{12}, q_{23})}
    \end{align*}
    
    \item $\Sigma = \Set{a,b}$
    \item $\delta$ is given by $\forall (r_1, r_2) \in Q \; and \; each \; a \in \Sigma$, \\ let $\delta((r_1, r_2), a) = (\delta(r_1, a), \delta(r_2, a))$.  Computing this table gives the following,
    
\begin{figure}[H]
\centering

\[
\begin{array}{c||c|c}
         & a & b \\ \hline \hline 
        (q_{11},q_{21}) & (q_{12},q_{21}) & (q_{11},q_{22}) \\ \hline 
        (q_{11},q_{22}) & (q_{12},q_{22}) & (q_{11},q_{23}) \\ \hline 
        (q_{11},q_{23}) & (q_{12},q_{23}) & (q_{11},q_{23}) \\ \hline 
        (q_{12},q_{21}) & (q_{11},q_{21}) & (q_{12},q_{22}) \\ \hline 
        (q_{12},q_{22}) & (q_{11},q_{22}) & (q_{12},q_{23}) \\ \hline 
        (q_{12},q_{23}) & (q_{11},q_{23}) & (q_{12},q_{23}) \\ \hline 
\end{array}
\]

\caption{}
\label{fig:fdsfsdfdsfd}
\end{figure}

    
    \item $q_0$ is the pair $(q_1, q_2) = (q_{11}, q_{21})$
    \item $F$ is $F_1 \times F_2$
\end{enumerate}

As a state diagram, we have,

% \includegraphics[width=\textwidth]{test.png}

\begin{figure}[H]
    \centering
    \includegraphics[scale=.1, angle=90]{1b.jpg}
    \caption{State diagram for DFA recognizing $A_1 \cap A_2$}
    \label{fig:my_label}
\end{figure}

% 1C ------------------------------

\subsubsection{Use construction in the proof of Theorem 1.45 to give the state diagram of NFA recognizing the union of the languages $A$ and $B$ over the alphabet $\Sigma = \Set{0,1}$}

\begin{align*}
    A &= \Set{w \given w \; \text{begins with a 1 and ends with a 0}} \\
    B &= \Set{w \given w \; \text{contains at least three 0's}} \\
\end{align*}

First, we give the NFA, $N_1$, recognizing the language $A$, where $N_1 = (Q_1, \Sigma, \delta_1, q_1, F_1)$.

\begin{enumerate}
    \item $Q_1 = \Set{q_{11}, q_{12}, q_{13}}$
    \item $\Sigma = \Set{0,1}$
    \item $\delta_1$ is given by,
    
\begin{figure}[H]
\centering

\[
\begin{array}{c||c|c|c}
         & 0 & 1 & \epsilon \\ \hline \hline 
        q_{11} & \emptyset & \Set{q_{12}} & \emptyset \\ \hline 
        q_{12} & \Set{q_{13}} & \Set{q_{12}} & \emptyset \\ \hline 
        q_{13} & \Set{q_{13}} & \Set{q_{12}} & \emptyset \\ \hline 
\end{array}
\]

\caption{}
\label{fig:mylabel}
\end{figure}

    
    \item Start state is $q_1 = q_{11}$
    \item Set of accept states is $F_1 = \Set{q_{13}}$
\end{enumerate}

\paragraph{}

Second, we give the NFA recognizing language $B$.  That is, $N_2 = (Q_2, \Sigma, \delta_2, q_2, F_2)$.

\begin{enumerate}
    \item $Q_2 = \Set{q_{21},q_{22},q_{23},q_{24}}$
    \item $\Sigma = \Set{0,1}$
    \item $\delta_2$ given by, 
    
\begin{figure}[H]
\centering

\[
\begin{array}{c||c|c|c}
         & 0 & 1 & \epsilon \\ \hline \hline 
        q_{21} & \Set{q_{22}} & \Set{q_{21}} & \emptyset \\ \hline 
        q_{22} & \Set{q_{23}} & \Set{q_{22}} & \emptyset \\ \hline 
        q_{23} & \Set{q_{24}} & \Set{q_{23}} & \emptyset \\ \hline 
        q_{24} & \Set{q_{24}} & \Set{q_{24}} & \emptyset \\ \hline 
\end{array}
\]

\caption{}
\label{fig:mylabel}
\end{figure}

    
    \item Start state is $q_2 = q_{21}$
    \item Set of accept states is $F_2 = \Set{q_{24}}$
\end{enumerate}

Third, we construct $N$ to recognize $A \cup B$ over alphabet $\Sigma = \Set{0,1}$, where $N = (Q, \Sigma, \delta, q_0, F)$.

\begin{enumerate}
    \item $Q = \Set{q_0} \cup Q_1 \cup Q_2 = \Set{q_0,q_{11},q_{12},q_{13},q_{21},q_{22},q_{23},q_{24}}$
    \item State $q_0$ is the start state of $N$
    \item Set of accept states $F = F_1 \cup F_2 = \Set{q_{13},q_{24}}$
    \item $\delta$ is given by,
    
\begin{figure}[H]
\centering

\[
\begin{array}{c||c|c|c}
         & 0 & 1 & \epsilon \\ \hline \hline 
        q_0 & \emptyset & \emptyset & \Set{q_{11},q_{21}} \\ \hline 
        q_{11} & \emptyset & \Set{q_{12}} & \emptyset \\ \hline 
        q_{12} & \Set{q_{13}} & \Set{q_{12}} & \emptyset \\ \hline 
        q_{13} & \Set{q_{13}} & \Set{q_{12}} & \emptyset \\ \hline 
        q_{21} & \Set{q_{22}} & \Set{q_{21}} & \emptyset \\ \hline 
        q_{22} & \Set{q_{23}} & \Set{q_{22}} & \emptyset \\ \hline 
        q_{23} & \Set{q_{24}} & \Set{q_{23}} & \emptyset \\ \hline 
        q_{24} & \Set{q_{24}} & \Set{q_{24}} & \emptyset \\ \hline 
\end{array}
\]

\caption{}
\label{fig:mylabel}
\end{figure}
    
\end{enumerate}

\paragraph{}

This concludes our construction of $N$ recognizing $A \cup B$ over alphabet $\Sigma = \Set{0,1}$.  As a state diagram, $N$ is,

\begin{figure}[H]
    \centering
    \includegraphics[scale=.1, angle=90]{1c.jpg}
    \caption{State diagram for question 1(c) with $N$ recognizing $A \cup B$ over alphabet $\Sigma = \Set{0,1}$.}
    \label{fig:my_label}
\end{figure}

% 1D ------------------------------------------------

\subsubsection{Use the construction in the proof of Theorem 1.47 to give the state diagram of NFA recognizing the concatenation of languages $A$ and $B$ over the alphabet $\Sigma = \Set{0,1}$}

\begin{align*}
    A &= \Set{w \given w \; \text{contains at least three 0's}} \\
    B &= \emptyset \\
\end{align*}

First, we give the NFA of languages $A$.  That is, $N_1 = (Q_1, \Sigma, \delta_1, q_0, F_1)$.

\begin{enumerate}
    \item $Q_1 = \Set{q_0,q_1,q_2,q_3}$
    \item $\Sigma = \Set{0,1}$
    \item $\delta_1$ is given by,
    
\begin{figure}[H]
\centering

\[
\begin{array}{c||c|c|c}
         & 0 & 1 & \epsilon \\ \hline \hline 
        q_0 & \Set{q_1} & \emptyset & \emptyset \\ \hline 
        q_1 & \Set{q_2} & \emptyset & \emptyset \\ \hline 
        q_2 & \Set{q_3} & \emptyset & \emptyset \\ \hline 
        q_3 & \Set{q_3} & \Set{q_3} & \emptyset \\ \hline 
\end{array}
\]

\caption{}
\label{fig:mylabel}
\end{figure}

    
    \item Start state is $q_0$
    \item Set of accept states is $F_1 = \Set{q_3}$
\end{enumerate}

\paragraph{}

Second, we give the NFA recognizing $B$.  That is, $N_2 = (\Set{\emptyset}, \Set{0,1}, \delta_2, \emptyset, \Set{\emptyset})$ (shorthand notation here for brevity).  $\delta_2$ is given by the empty table (since $Q_2 = \Set{\emptyset}$).

We construct $N$ recognizing the concatenation of $A$ and $B$ (i.e. $A \circ B$) over the alphabet $\Sigma = \Set{0,1}$.  That is, $N = (Q, \Sigma, \delta, q_{start}, F)$.

\begin{enumerate}
    \item $Q = Q_1 \cup Q_2 = Q_1 \cup \emptyset = Q_1 = \Set{q_0, q_1, q_2, q_3}$
    \item $\Sigma = \Set{0,1}$
    \item $\delta$ is given by,
    
\begin{figure}[H]
\centering

\[
\begin{array}{c||c|c|c}
         & 0 & 1 & \epsilon \\ \hline \hline 
        q_0 & \Set{q_1} & \emptyset & \emptyset \\ \hline 
        q_1 & \Set{q_2} & \emptyset & \emptyset \\ \hline 
        q_2 & \Set{q_3} & \emptyset & \emptyset \\ \hline 
        q_3 & \Set{q_3} & \Set{q_3} & \emptyset \\ \hline 
\end{array}
\]

\caption{}
\label{fig:mylabel}
\end{figure}

    
    \item Start state is $q_{start} = q_0$
    \item Set of accept states is same as $F_2$, that is, $F = F_2 = \Set{\emptyset}$
\end{enumerate}

\paragraph{}

This concludes the construction of $N$ recognizing $A \circ B$ over alphabet $\Sigma = \Set{0,1}$.  As a state diagram, $N$ is,

\begin{figure}[H]
    \centering
    \includegraphics[scale=.1, angle=90]{1d.jpg}
    \caption{Caption}
    \label{fig:my_label}
\end{figure}

% 2 --------------------------------------------

\subsection{Let $N = (\Set{q_1,q_2,q_3}, \Set{a,b}, \delta, q_1, \Set{q_2})$ be an NFA where $\delta$ is given by the table below.  Use the construction given in Theorem 1.39 to convert it to an equivalent deterministic finite automaton.}

\begin{figure}[H]
\centering

\[
\begin{array}{c||c|c|c}
         & a & b & \epsilon \\ \hline \hline 
        q_1 & \Set{q_3} & \emptyset & \Set{q_2} \\ \hline 
        q_2 & \Set{q_1} & \emptyset & \emptyset \\ \hline 
        q_3 & \Set{q_2} & \Set{q_2,q_3} & \emptyset \\ \hline 
\end{array}
\]

\caption{NFA $N$'s transition function $\delta$ for question 2.}
\label{fig:mylabel}
\end{figure}

We construct DFA $D = (Q', \Sigma, \delta', q_1, F')$.  

\begin{enumerate}
    \item $Q' = P(Q)$.  That is, every state of $D$ is a set of states of $N$.  Thus, $Q' = \Set{\emptyset, \Set{q_1}, \Set{q_2}, \Set{q_3}, \Set{q_1,q_2}, \Set{q_2,q_3}, \Set{q_1,q_3}, \Set{q_1,q_2,q_3}}$
    \item The language $\Sigma$ is the same in $D$ as in $N$.
    \item $D$'s transition function $\delta'$ is given as,
    
\begin{figure}[H]
\centering

\[
\begin{array}{c||c|c}
         & a & b \\ \hline \hline 
        \emptyset & \emptyset & \emptyset \\ \hline 
        \Set{q1} & \Set{q3} & \emptyset \\ \hline 
        \Set{q2} & \Set{q1,q2} & \emptyset \\ \hline 
        \Set{q3} & \Set{q2} & \Set{q2,q3} \\ \hline 
        \Set{q1,q2} & \Set{q1,q2,q3} & \emptyset \\ \hline 
        \Set{q2,q3} & \Set{q1,q2} & \Set{q2,q3} \\ \hline 
        \Set{q1,q3} & \Set{q2,q3} & \Set{q2,q3} \\ \hline 
        \Set{q1,q2,q3} & \Set{q1,q2,q3} & \Set{q2,q3} \\ \hline 
\end{array}
\]

\caption{}
\label{fig:mylabel}
\end{figure}

    
    \item Start state of $D$ is $E(\Set{q_1})$ set of states reachable from $q_1$ by travelling along $\epsilon$ arrows, plus $q_1$ itself.  So, $E(\Set{q_1}) = E(\Set{q_1,q_2})$.
    \item Set of accept states for $D$, $F'$ are those containing $N$'s accept states (i.e. $q_2$), thus, $F' = \Set{\Set{q_2}, \Set{q_1,q_2}, \Set{q_2,q_3}, \Set{q_1,q_2,q_3}}$.
\end{enumerate}

\paragraph{}

This concludes our construction of an equivalent DFA, as desired (state diagram omitted, as question 2 does not specifically request one).  

\subsection{Convert the regular expression $a^+ \cup (ab)^+$ where $\Sigma = \Set{a,b}$ using the construction in Lemma 1.55.}

Recall that if a language is described by a regular expression, then it is regular (by Lemma 1.55).  We know, by Corollary 1.40, if an NFA recognizes $A$, then $A$ is regular.  It suffices to convert $R$, given as $R = a^+ \cup (ab)^+$ into an NFA recognizing $A$. \\

Building up our NFA on the regular expression $R$, and using the operator precedence (i.e. star, concatenation, then union), we arrive at NFA $N = (Q, \Sigma, \delta, q_0, F)$, recognizing $L(R)$. \\

\begin{enumerate}
    \item $Q = \Set{q_0,q_1,q_2,q_3,q_4,q_5,q_6}$
    \item $\Sigma = \Set{a,b}$, (also note that since $N$ is an NFA, the empty string $\epsilon$ is part of the language $\Sigma_{\epsilon} = \Set{a,b,\epsilon}$)
    \item $\delta$ is given by,
    
\begin{figure}[H]
\centering

\[
\begin{array}{c||c|c|c}
         & a & b & \epsilon \\ \hline \hline 
        q_0 & \emptyset & \emptyset & \Set{q_1,q_3} \\ \hline 
        q_1 & \Set{2} & \emptyset & \emptyset \\ \hline 
        2 & \emptyset & \emptyset & \Set{q_1} \\ \hline 
        q_3 & \Set{q_4} & \emptyset & \emptyset \\ \hline 
        q_4 & \emptyset & \emptyset & \Set{q_5} \\ \hline 
        q_5 & \emptyset & \Set{q_6} & \emptyset \\ \hline 
        q_6 & \emptyset & \emptyset & \Set{q_3} \\ \hline 
\end{array}
\]

\caption{}
\label{fig:3}
\end{figure}

    
    \item Start state is $q_0$
    \item Set of accept states $F = \Set{q_2,q_6}$
\end{enumerate}

\paragraph{}

This concludes our conversion of the regular expression into an NFA.  For convenience, I have included a state diagram (below) of the NFA described formally above in \ref{fig:3}, as well as some intermediate steps (for $a^+$ and $(ab)^+$).  

\begin{figure}[H]
    \centering
    \includegraphics[scale=.1, angle=90]{3.jpg}
    \caption{State diagram of NFA (i.e. $N$) for question 3, and some work showing intermediate steps for $a^+$ and $(ab)^+$ (included for convenience).  }
    \label{fig:my_label}
\end{figure}

\end{document}
