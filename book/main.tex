\documentclass{article}
% \usepackage[utf8]{inputenc}

% usage: from Options > configure TexStudio > Build > add user command
% makeindex -s nomencl.ist -t %.nlg -o %.nls %.nlo
% Now run pdflatex -> Make Nomenclature -> pdflatex
\usepackage{nomencl}
\makenomenclature

\usepackage{tikz}
\usetikzlibrary{automata,positioning, shapes.geometric}
\usepackage{subfig}
\usepackage{pgf}
\usepackage{float}
\usepackage{titlesec}
\usepackage[nottoc]{tocbibind}
\usepackage{amsmath}
\usepackage{hyperref}
\usepackage{enumitem}
\usepackage{listings}
\usepackage{xcolor}
\usepackage{amsmath,amssymb,amsthm}
\usepackage{graphicx}
\graphicspath{./images/}



% set builder notation utility function, usage: \Set{ x\in A \given x^2 \geq 3 }
\usepackage{mathtools}
\newcommand\SetSymbol[1][]{\nonscript\:#1\vert\allowbreak\nonscript\:\mathopen{}}
\providecommand\given{} % to make it exist
\DeclarePairedDelimiterX\Set[1]\{\}{\renewcommand\given{\SetSymbol[\delimsize]}#1}

% some stuff for C++
\lstset { %
	language=C++,
	backgroundcolor=\color{black!5}, % set backgroundcolor
	basicstyle=\footnotesize,% basic font setting
	breaklines=true,
	postbreak=\mbox{\textcolor{red}{$\hookrightarrow$}\space},
	frame=single,
	columns=fullflexible
}

% TABLE OF CONTENTS, MANUALLY SET NESTING DEPTH
%\setcounter{tocdepth}{1} % show sections
%\setcounter{tocdepth}{2} % show subsections
%\setcounter{tocdepth}{3} % show subsubsections
%\setcounter{tocdepth}{4} % show paragraphs
%\setcounter{tocdepth}{5} % show subparagraphs

% VARIABLES
% --- Hide title page
%\newif\ifhidetitle
%\hidetitletrue

% --- make subsections have lettering (e.g. 1.a, 1.b)
% \renewcommand{\thesubsection}{\thesection.\alph{subsection}}

% --- make subsections have arabic numbering (e.g. 1.1, 1.2, ...)
\renewcommand{\thesubsection}{\arabic{subsection}}

% --- End of proof black square (remove if you want default hollow white square)
\renewcommand{\qedsymbol}{$\blacksquare$}

\titlespacing*{\section}
{0pt}{5.5ex plus 1ex minus .2ex}{4.3ex plus .2ex}
\titlespacing*{\subsection}
{0pt}{5.5ex plus 1ex minus .2ex}{4.3ex plus .2ex}


\begin{document}

% --- title, author, date all on title page, use \\ within a {} to linebreak
\title{Competitive Programming\\Code and Math Library}
\author{Cody Barnson}
\date{November 2018}

% TITLE PAGE, example of variable usage
%\ifhidetitle
%\else

% --- no page numbers for title page and table of contents
\pagenumbering{gobble}
\maketitle
% \newpage

% --- uncomment to show the table of contents
\tableofcontents
%\listoftables
%\listoffigures
\newpage

% --- this 'fi' belongs to the \ifhidetitle \else block above
%\fi

% --- normal page numbers starting from here, can also do roman
% \renewcommand{\thesubsubsection}{\thesubsection.\alph{subsubsection}}
\pagenumbering{arabic}

\makeatletter
\newcommand{\vo}{\vec{}}


% ====== START HERE

\section*{Geometry}

\subsection{Basics}

\subsubsection{Cross product}

Cross product, in 2D, 

\begin{align*}
    \vec{v} \times \vec{w} = \lVert \vec{v} \lVert \lVert \vec{w} \lVert sin \theta
\end{align*}

\section*{LaTeX}

\subsection{Examples}

\paragraph{Matrices}

\begin{figure}[H]
	\centering
	\[
		\begin{pmatrix}1 & 2 & -3 \\ 4 & 0 & 1\end{pmatrix}
	\]
	\caption{pmatrix}
	\label{fig:pmatrix}
\end{figure}

\end{document}
